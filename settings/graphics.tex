% Packages
\usepackage{xcolor}

%%%%%%%%%%%%%%%%%%%%%%%%
%        COLORS        %
%%%%%%%%%%%%%%%%%%%%%%%%

% Normal colors
\definecolor{xred}{HTML}{BD4242}
\definecolor{xblue}{HTML}{4268BD}
\definecolor{xgreen}{HTML}{52B256}
\definecolor{xpurple}{HTML}{7F52B2}
\definecolor{xorange}{HTML}{FD9337}
\definecolor{xdotted}{HTML}{999999}
\definecolor{xgray}{HTML}{777777}
\definecolor{xcyan}{HTML}{80F5DC}
\definecolor{xpink}{HTML}{F690EA}
\definecolor{xgrayblue}{HTML}{49B095}
\definecolor{xgraycyan}{HTML}{5AA1B9}

% Dark colors
\colorlet{xdarkred}{red!85!black}
\colorlet{xdarkblue}{xblue!85!black}
\colorlet{xdarkgreen}{xgreen!85!black}
\colorlet{xdarkpurple}{xpurple!85!black}
\colorlet{xdarkorange}{xorange!85!black}
\colorlet{xdarkcyan}{xcyan!85!black}

% Very dark colors
\colorlet{xverydarkblue}{xblue!50!black}

% Document-specific colors
\colorlet{normaltextcolor}{black}
\colorlet{figtextcolor}{xblue}

% Enumerated colors
\colorlet{xcol0}{black}
\colorlet{xcol1}{xred}
\colorlet{xcol2}{xblue}
\colorlet{xcol3}{xgreen}
\colorlet{xcol4}{xpurple}
\colorlet{xcol5}{xorange}
\colorlet{xcol6}{xcyan}
\colorlet{xcol7}{xpink!75!black}

%%%%%%%%%%%%%%%%%%%%%
%        PGF        %
%%%%%%%%%%%%%%%%%%%%%

\usepackage{pgfplots}
\usepgfplotslibrary{fillbetween}%, colormaps, colorbrewer, patchplots}
\pgfplotsset{
  compat=1.16,
  %% Styles %%
  xyplane/.style = {
    axis x line=middle,
    axis y line=middle,
    xlabel=$x$,
    ylabel=$y$,
    every axis x label/.style={at={(ticklabel* cs:1.02)}, anchor=west},
    every axis y label/.style={at={(ticklabel* cs:1.02)}, anchor=south},
    axis line style={stealth-stealth, very thick, black!65},
    label style={font=\large},
    tick label style={font=\large},
    samples=100,
    xmin=-5, xmax=5,
    ymin=-5, ymax=5,
    grid=both,
    major grid style={black!15},
    minor grid style={black!10},
    xticklabels={,},
    yticklabels={,},
  },
  xynogrid/.style = {
    xyplane,
    major grid style={opacity=0},
    minor grid style={opacity=0},
  },
  xyempty/.style = {
    xyplane,
    axis line style = {draw=none},
    tick style = {draw=none},
    xlabel = {},
    ylabel = {},
    major grid style = {draw=black!0},
  },
  hyperplane1D/.style = {thick, #1},
  %% Function plots
  function/.style = {
    ultra thick, draw=#1,
  },
  filledfunction/.style = {
    function={#1}, opacity=1, fill=#1, fill opacity=0.3,
  },
}

%%%%%%%%%%%%%%%%%%%%%%
%        TikZ        %
%%%%%%%%%%%%%%%%%%%%%%

\usepackage{tikz, tikz-3dplot}
\usetikzlibrary{calc, shapes, intersections}
\tikzset{
  %% Styles %%
  vector/.style = {#1, ultra thick, -stealth, cap=round},
  pics/ruler/.style n args = {5}{
      code = {
          % Parameters
          \pgfmathsetmacro{\width}{#1}
          \pgfmathsetmacro{\height}{#2}
          \pgfmathsetmacro{\yMaj}{\height/3}
          \pgfmathsetmacro{\yMid}{\yMaj*0.7}
          \pgfmathsetmacro{\yMin}{\yMaj*0.5}
          \pgfmathsetmacro{\dxMaj}{#3}
          \pgfmathsetmacro{\dxMin}{\dxMaj*0.1}
          \pgfmathsetmacro{\maxMajGrad}{ceil((\width-2)/\dxMaj)}
          \pgfmathsetmacro{\maxMidGrad}{\maxMajGrad-0.5}
          \pgfmathsetmacro{\maxMinGrad}{(\maxMajGrad-1)*10}

          % Main rectangle
          \draw[thick, fill=#4] (0,0) rectangle (\width,\height);

          % Graduations
          % Major
          \foreach \x [count=\k from 0] in {1,2,...,\maxMajGrad}
              \draw[thick] ({\x*\dxMaj}, \height) -- ++(0.0,-\yMaj) node[below] (g\k) {$\k$};
          % Middle
          \foreach \y in {1.5,2.5,...,\maxMidGrad}
              \draw[thick] ({\y*\dxMaj}, \height) -- ++(0.0,-\yMid);
          % Minor
          \foreach \z in {1,2,...,\maxMinGrad} 
              \draw[thin] ({\dxMaj+\z*\dxMin}, \height) -- ++(0.0,-\yMin);

          % units label
          \node[below of=g0, font=\small, yshift={11*\height}] {#5};
      }
  }
}
\newcommand{\rulerTwoD}[6]{
  \pgfmathsetmacro{\Vx}{#1};
  \pgfmathsetmacro{\Vy}{#2};
  \pgfmathsetmacro{\a}{#3};
  \pgfmathsetmacro{\scale}{3*\a/sqrt(\Vx*\Vx+\Vy*\Vy)};
  \foreach \b in {-#4,...,#4}
    \addplot[very thick, #5] {-(\Vx/\Vy)*x+(\b/\a)};
}
