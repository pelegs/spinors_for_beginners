\usepackage{enumitem}
\usepackage{fontawesome}

% Bold text itemize/enumerate
% Taken from an answer to the TeX StackExchange question #266225
\newenvironment{descitemize}
{\begin{description}[leftmargin=*, before=\let\makelabel\descitemlabel]}
{\end{description}}
\newcommand{\descitemlabel}[1]{\textbullet\ \textbf{#1}:}

\usepackage[most]{tcolorbox}

\tikzset{
	second box/.style={
		anchor=east,
		text=white,
		rounded corners,
		fill=#1,
		xshift=-4mm,
	},
}

\tikzset{tcbreak/.style = {fill=white, thick, rounded corners, draw=#1, text=#1, font=\bfseries}}

\tcbset{
	common/.style n args={2}{
		colframe={#1},
		colback={#1!5},
		colbacktitle={#1},
		lower separated=false,
		coltitle=white,
		boxrule=1pt,
		fonttitle=\bfseries,
		enhanced,
		breakable,
		top=8pt,
		before skip=1em,
		after skip=2em,
		attach boxed title to top left={
			yshift=-0.25cm,
			xshift=0.38cm,
		},
		boxed title style={
			boxrule=0pt,
			colframe=white,
			arc=5pt,
			outer arc=4pt,
		},
		separator sign={~~},
		% -- overlays based on the following TeX SE answer: https://tex.stackexchange.com/a/545324/162087
		overlay unbroken={
			\node[text=white, align=right, rounded corners, fill=#1, xshift=-4mm, minimum height=6mm, anchor=east] at (frame.south east) {#2};
		},
		overlay first={
			\draw[thick, #1, decoration={coil, amplitude=0.5mm}, decorate]
				(frame.south west) -- (frame.south east);
			\path[font=\small\itshape] (frame.south) node [tcbreak={#1}] (cont) {continues in the next page \faHandORight};
		},
		overlay middle={%
			% upper break
			\draw[thick, #1, decoration={coil, amplitude=0.5mm}, decorate]
				(frame.north west) -- (frame.north east);
			\path[font=\small\itshape] (frame.north) node [tcbreak={#1}] (cont) {\faHandORight continues from the previous page};

			% lower break
			\draw[thick, #1, decoration={coil, amplitude=0.5mm}, decorate]
				(frame.south west) -- (frame.south east);
			\path[font=\small\itshape] (frame.south) node [tcbreak={#1}] (cont) {continues in the next page \faHandORight};
		},
		overlay last={
			\node[text=white, align=right, rounded corners, fill=#1, xshift=-4mm, minimum height=6mm, anchor=east] at (frame.south east) {#2};
			\draw[thick, #1, decoration={coil, amplitude=0.5mm}, decorate]
				(frame.north west) -- (frame.north east);
			\path[font=\small\itshape] (frame.north) node [tcbreak={#1}] (cont) {\faHandORight continues from the previous page};
		}
	},
	defstyle/.style={
		common={xpurple}{$\bm{\pi}$},
	},
	theoremstyle/.style={
		common={xgraycyan}{$\multimapdotbothA$},
	},
	lemmastyle/.style={
		common={xgrayblue}{$\multimap$},
	},
	proofstyle/.style={
		common={xgreen}{\textbf{QED}},
	},
	examplestyle/.style={
		common={xblue}{\faStar},
	},
	notestyle/.style={
		common={xred}{\textbf{!}},
	},
	challengestyle/.style={
		common={xorange}{\textbf{?}},
	},
	quotestyle/.style={
		common={gray}{\textbf{"}},
	},
}

\newtcbtheorem[auto counter, number within=chapter]{definition}{Definition}{defstyle}{def}
\newtcbtheorem[auto counter, number within=chapter]{theorem}{Theorem}{theoremstyle}{theorem}
\newtcbtheorem[auto counter, number within=chapter]{lemma}{Lemma}{lemmastyle}{lemma}
\newtcbtheorem[auto counter, number within=chapter]{proof}{Proof}{proofstyle}{proof}
\newtcbtheorem[auto counter, number within=chapter]{example}{Example}{examplestyle}{example}
\newtcbtheorem[auto counter, number within=chapter]{note}{Note}{notestyle}{note}
\newtcbtheorem[auto counter, number within=chapter]{challenge}{Challenge}{challengestyle}{challenge}
\newtcbtheorem[]{nquote}{Quote}{quotestyle}{quote}
