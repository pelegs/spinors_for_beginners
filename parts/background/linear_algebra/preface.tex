\section{Preface}
\newthought{The goal of this chapter} is not to teach you, the reader, linear algebra from scratch - nor to be a thorough source of information on the topic. Rather, my aim is to introduce important \enquote{advanced} concepts for those who took a basic linear algebra course as part of an undergraduate university program. These concpets should help you gain a basic knowledge of the topics needed for understanding the rest of the background material, as well as the topic of spinors itself.

My approach to teching topics in linear algebra - and in mathematics as a whole - is to first build an intuition and only then formalize and generalize the ideas as needed. In my personal experiences, when I was studying linear algebra I completely failed to understand it (and indeed, failed the course) until it \enquote{clicked} for me in regards to 2- and 3-dimensional real spaces, i.e. - visible geometry. After that I didn't even have to study for exams anymore, as everything became clear enough to grasp and develop on the spot even during an exam (except for later, more advances concepts). That is why, for example, I absolutely adore courses and study materials of the topic\sidenote{And other mathematical topics as well.} which use animation, such as \textit{3Blue1Brown} great video essay series \href{https://www.3blue1brown.com/topics/linear-algebra}{Essence of linear algebra}\sidenote{Temporary sidenote which should become a citation for the mentioned 3B1B video series}.

There are very few proofs in this chapter, and those that are shown are not completely rigorous. For more in-depth materials, see the last section (further read). With that out of the way - let's begin!
\newpage
