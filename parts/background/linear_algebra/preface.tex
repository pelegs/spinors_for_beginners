\section{Preface}
\newthought{The goal of this chapter} is not to teach you, the reader, linear algebra from scratch - nor to be a thorough source of information on the topic. Rather, my aim is to overview the topic in such a way that new-comers, as well as those who studied linear algebra in an undergraduate university course, will gain the important insights of the topic needed for understanding the rest of the background material and spinors as well.

Instead of teaching the topic from the ground-up, like mathematicians tend to do\sidenote{In my view, courses that \enquote{build} linear algebra step-by-step give the students good knowledge of the structures of vector spaces, but tend to miss the intuitive view of what these structures can \textit{do}. This is exactly the difference between the \enquote{pure} mathematics of the mathematician and the mathematics as a tool of the scientist.}, I prefer to stick to the geometric interpretation of the vector spaces $\Rs[2]$ and $\Rs[3]$ (and to a lesser extent $\Rs[n]$ in general). These interpretations can be visualized relatively easily, and thus help in setting up the needed intuition in the student's mind, which becomes handy when the topic turns to more abstract constructs (such as for example vector spaces of matrices or functions).

In my personal experiences, when I was studying the topic I completely failed to understand it (and indeed, failed the courses I took) until it \enquote{clicked} for me in regards to 2- and 3-dimensional real spaces, i.e. - visible geometry. Then I didn't even have to study for exams anymore, as everything became clear enough to grasp and develop on the spot even during an exam (except for later, more advances concepts). That is why, for example, I absolutely adore courses and study materials of the topic\sidenote{And other mathematical topics as well.} which use animation, such as \textit{3Blue1Brown} great video essay series \href{https://www.3blue1brown.com/topics/linear-algebra}{Essence of linear algebra}\sidenote{Temporary sidenote which should become a citation for the mentioned 3B1B video series}.

There are very few proofs in this chapter, and those that are shown are not completely rigorous. For more in-depth materials, see the last section (further read). With that out of the way - let's begin!
