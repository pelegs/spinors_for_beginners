\section{Change of Coordinates}
\newthought{In introductory linear algebra courses} you should have learned about change of coordinate systems: a coordinate system is just another name for a basis set of whatever vector space is used (in this section it's $\Rs[n]$). A change of coordinate system is the transformation of vectors from being represented in one basis set $B=\left\{\eb{1},\eb{2},\dots,\eb{n}\right\}$ to being represented in another basis set $\tilde{B}=\left\{\ebc{1},\ebc{2},\dots,\ebc{n}\right\}$. Since such transformations are linear they are commonly represented in a matrix form.

In this section we will discuss \textit{how} vectors and their components transform under change of basis sets. These transformations can be quite confusing, so I color coded the equations consistently as a visual guide. In addition, the $\Rs[2]$ case is intoduced first, before giving the generalized form for $\Rs[n]$. \sidenote{This section is heavily based on Eigenchris' video series \href{https://www.youtube.com}{Tensors for Beginners}.}

Before we do this, there is one important idea to understand: vectors themselves do not change under change of basis - their magnitude and orientation stay the same no matter how \textit{we} measure them. The component of a vector are a way to describe (or \enquote{measure}) the vector using a certain basis set, so it should not be surprising that they change under change of basis.

With that in mind, let us look at an example of a basis set change in $\Rs[2]$.

\subsection{Change of basis set in $\Rs[2]$}
Suppose we use the standard basis set to represent $\Rs[2]$:
\begin{equation}
    \color{xdarkblue}
    B = \left\{\eb{1},\eb{2}\right\} = \left\{\colvec{1;0},\colvec{0;1}\right\},
    \color{black}
    \label{eq:std_basis_set_R2}
\end{equation}
and we want to change our coordinate system to use the following basis set:
\begin{equation}
    \color{xdarkred}
    \tilde{B} = \left\{\ebc{1},\ebc{2}\right\} = \left\{\colvec{2;1},\colvec{-\frac{1}{2};\frac{1}{4}}\right\}.
    \color{black}
    \label{eq:transformed_basis_set_R2}
\end{equation}
(the vectors composing the two basis sets are shown in \autoref{fig:two_basis_sets_R2})

\begin{marginfigure}[0\baselineskip]
    \begin{center} 
        \begin{tikzpicture}
            \begin{axis}[
                xynoaxes,
                width=6cm, height=6cm,
                xmin=-0.75, xmax=2.5,
                ymin=-0.75, ymax=2.5,
                xtick={0,1,2}, extra x ticks=0,
                ytick={0,1,2}, extra y ticks=0,
                major grid style={dotted, draw=black!50},
            ]
                \draw[vector={xdarkblue}] (0,0) -- (1,0) node[pos=1.2] {$\eb{1}$};
                \draw[vector={xdarkblue}] (0,0) -- (0,1) node[pos=1.2] {$\eb{2}$};
                \draw[vector={xdarkred}] (0,0) -- (2,1) node[pos=1.075] {$\ebc{1}$};
                \draw[vector={xdarkred}] (0,0) -- (-0.5,0.25) node[pos=1.3] {$\ebc{2}$};
            \end{axis}
        \end{tikzpicture}
    \end{center}
    \caption{The standard basis set $\color{xdarkblue}{B}$ and a new basis set $\color{xdarkred}{\tilde{B}}$ shown together.}
    \label{fig:two_basis_sets_R2}
\end{marginfigure}

The transformation between the basis sets can be represented in matrix form:

\vspace{1.5em}
\begin{equation}
    \Forw =
    \begin{bmatrix}
        \tikzmark{ebc1}{2} & \tikzmark{ebc2}{-\frac{1}{2}} \\
        1 & \frac{1}{4}
    \end{bmatrix}.
    \label{eq:forward_trans}
\end{equation}
\tikz[overlay, remember picture]{
    \node[xdarkred] (ebc1txt) at ($(pic cs:ebc1)+(4pt,1cm)$) {$\ebc{1}$};
    \node[xdarkred] (ebc2txt) at ($(pic cs:ebc2)+(4pt,1cm)$) {$\ebc{2}$};
    \draw[-stealth, xdarkred] (ebc1txt) -- ($(pic cs:ebc1)+(4pt,10pt)$);
    \draw[-stealth, xdarkred] (ebc2txt) -- ($(pic cs:ebc2)+(4pt,10pt)$);
}

\begin{note}{The components of $\Forw$}{}
    The columns of $\Forw$ are exactly the two vectors of $\newB$ because $\oldB$ is the standard basis. The transformation matrix is not that obvious between any two basis sets.
\end{note}

Now, if we want to transform $\ebr{1}$ and $\ebr{2}$ into $\ebcr{1}$ and $\ebcr{2}$ using $\Forw$, we simply multiply each of the vectors by $\Forw$:
\begin{equation}
    \begin{aligned}
        \ebcr{1} &= \Forw \ebr{1},\\\nonumber
        \ebcr{2} &= \Forw \ebr{2}.
    \end{aligned}
    \label{eq:transforming_ebs}
\end{equation}

To make \autoref{eq:transforming_ebs} more concise, we can collect the two vectors into a matrix disguised as a row vecor:
\begin{equation}
    \color{xdarkblue}
    \eb{} =
    \begin{bmatrix}
        1 & 0\\
        0 & 1
    \end{bmatrix} = \rowvec{\eb{1},\eb{2}}
    \color{black},
    \label{eq:eb_matrix}
\end{equation}

and find that \autoref{eq:transforming_ebs} can be written in vector-matrix notation as
\begin{equation}
    \color{xdarkred}\ebcr{} = \rowvec{\ebcr{1};\ebcr{2}}
    \color{black} =
    \color{xdarkblue}\rowvec{\eb{1};\eb{2}}
    \color{black}
        \begin{bmatrix}
            2 & -\frac{1}{2}\\
            1 & \frac{1}{4}
        \end{bmatrix}
        = \rowvec{2\ebr{1}+\ebr{2};-\frac{1}{2}\ebr{1}+\frac{1}{4}\ebr{2}}.
    \label{eq:full_forward_trans}
\end{equation}

The reverse transformation can be calculated by applying the inverse transformation $\Backw$ on \autoref{eq:full_forward_trans}:
\begin{equation}
    \begin{aligned}
        \color{xdarkred}\rowvec{\ebc{1};\ebc{2}}
        \color{black}\Backw &=
        \left(
            \color{xdarkblue}\rowvec{\eb{1};\eb{2}}
            \color{black} \Forw
        \right)\Backw \nonumber\\
                            &= \color{xdarkblue}\rowvec{\eb{1};\eb{2}}
                               \color{black}\Forw\Backw \nonumber\\
                            &= \color{xdarkblue}\rowvec{\eb{1};\eb{2}}
                                \color{black}\bm{I}_{2} \nonumber\\
                            &= \color{xdarkblue}\rowvec{\eb{1};\eb{2}}.
    \end{aligned}
    \label{eq:eb_from_ebc}
\end{equation}
(where in our case $\Backw=\begin{bmatrix}\frac{1}{4} & \frac{1}{2}\\ -1 & 2\end{bmatrix}$)

\underline{To summarize}: given a set $\oldB$ of basis vectors in $\Rs[2]$, we can transform it into the basis vectors set $\newB$ by using the forward transformation $\Forw=\begin{bmatrix}F_{11} & F_{12} \\ F_{21} & F_{22}\end{bmatrix}$. To get back the original basis vectors from the transformed vectors we use the inverse transfomation $\Backw=\begin{bmatrix}F_{11} & F_{12} \\ F_{21} & F_{22}\end{bmatrix}^{-1}$ (if it exists - i.e. if $\Forw$ is invertible).

\subsection{The more general case: $\Rs[n]$}
In $\Rs[n]$ the transformations behave in a similar way: given the transformation rule that each new basis vector $\ebcr{i}\in\newB$ is a linear combination of the old basis vector set $\oldB$, i.e.
\begin{align*}
    \ebcr{1} &= F_{11}\ebr{1} + F_{12}\ebr{2} + \dots + F_{1n}\ebr{n},\\\nonumber
    \ebcr{2} &= F_{21}\ebr{1} + F_{22}\ebr{2} + \dots + F_{2n}\ebr{n},\\\nonumber
    \vdots &= \vdots \\ \nonumber
    \ebcr{n} &= F_{n1}\ebr{1} + F_{n2}\ebr{2} + \dots + F_{nn}\ebr{n},
    \label{eq:full_coords_trans_as_equations}
\end{align*}
we can write the transformation in matrix form as
\begin{equation}
    \color{xdarkred}\rowvec{\ebc{1},\ebc{2},\dots,\ebc{n}}
    \color{black}=
    \color{blue}\rowvec{\eb{1},\eb{2},\dots,\eb{n}}
    \color{black}\GNMatrix{F}{n}{n}.
    \label{eq:full_coords_trans}
\end{equation}

\autoref{eq:full_coords_trans} can be written in index notation, which shows us how each vector is transformed:
\begin{equation}
    \ebcr{j} = \displaystyle\sum_{k=1}^{n}F_{kj}\ebr{k}.
    \label{eq:coords_trans_per_vector}
\end{equation}
Similarly, the inverse operation is given by
\begin{align*}
    \ebr{1} &= F^{-1}_{11}\ebcr{1} + F^{-1}_{12}\ebcr{2} + \dots + F^{-1}_{1n}\ebcr{n},\\\nonumber
    \ebr{2} &= F^{-1}_{21}\ebcr{1} + F^{-1}_{22}\ebcr{2} + \dots + F^{-1}_{2n}\ebcr{n},\\\nonumber
    \vdots &= \vdots \\ \nonumber
    \ebr{n} &= F^{-1}_{n1}\ebcr{1} + F^{-1}_{n2}\ebcr{2} + \dots + F^{-1}_{nn}\ebcr{n},
    \label{eq:full_BACK_coords_trans_as_equations}
\end{align*}
with the matrix form being
\begin{equation}
    \color{xdarkblue}\rowvec{\eb{1},\eb{2},\dots,\eb{n}}
    \color{black}=
    \color{xdarkred}\rowvec{\ebc{1},\ebc{2},\dots,\ebc{n}}
    \color{black}\GNMatrix{F^{-1}}{n}{n},
    \label{eq:full_BACK_coords_trans}
\end{equation}
and the index notation being
\begin{equation}
    \ebr{i} = \displaystyle\sum_{j=1}^{n}F^{-1}_{ji}\ebcr{j}.
    \label{eq:coords_BACK_trans_per_vector}
\end{equation}
By subtituting \autoref{eq:coords_trans_per_vector} into \autoref{eq:coords_BACK_trans_per_vector}, we get
\begin{equation}
    \begin{aligned}
        \ebr{i} &= \displaystyle\sum_{j=1}^{n}F^{-1}_{ji}\ebcr{j}\\
                &= \displaystyle\sum_{j=1}^{n}F^{-1}_{ji}\left(\displaystyle\sum_{k=1}^{n}F_{kj}\ebr{k}\right)\\
                &= \displaystyle\sum_{k=1}^{n}\left(\tikzmark{scval1}\displaystyle\sum_{j=1}^{n}F_{kj}F^{-1}_{ji}\tikzmark{scval2}\right)\ebr{k}.
    \end{aligned}
    \label{eq:transformation_subtitution}
\end{equation}
\tikz[overlay, remember picture]{
    \draw[curly={xdarkgreen}{5pt}{9pt}] ($(pic cs:scval1)+(0,-10pt)$) -- ($(pic cs:scval2)+(0,-10pt)$) node[midway, below, yshift=-13pt] {this is just a number!}; 
}

Now, \autoref{eq:transformation_subtitution} simply tells us something we already know: each basis vector $\ebr{i}$ is equal to a linear combination of the \textit{same set} of basis vectors. This must mean that for $k=i$ the number in paranthesis is one, and for any other value of $k$ it is zero - i.e. it equals $\delta_{ik}$:
\begin{equation}
    \ebr{i}=\displaystyle\sum_{k=1}^{n}\delta_{ik}\ebr{k}.
    \label{eq:trans_delta}
\end{equation}

In turn, \autoref{eq:trans_delta} means that the matrix $\Forw\Backw=\bm{I}_{n}$, the identity matrix in $\Rs[n]$ - and thus the matrices $\Forw$ and $\Backw$ are eachother's inverses, exactly as we expect.

\subsection{Components transformation}
Now let us discuss how the \textit{components} of a vector transform under change of basis: suppose we have the vector $\vgreen$ as depicted in \autoref{fig:vec_in_2_basis_sets}. Using the standard basis set (depicted in blue) we can write $\vgreen=\ebr{1}+\ebr{2}$, or simply $\vgreen=\color{xdarkgreen}{\colvec{1;1}}$. However, by using the basis set $\newB=\left\{2\ebr{1};\ebr{2}\right\}$ (i.e. $\ebr{1}$ is scaled by $2$), $\vgreen$ is now $\vgreen=\frac{1}{2}\ebcr{1}+\ebcr{2}=\color{xdarkgreen}{\colvec{\frac{1}{2};1}}$.

\begin{marginfigure}[0\baselineskip]
    \begin{center} 
        \begin{tikzpicture}
            \begin{axis}[
                xynoaxes,
                width=6cm, height=6cm,
                xmin=-0.75, xmax=2.5,
                ymin=-0.75, ymax=2.5,
                xtick={0,1,2}, extra x ticks=0,
                ytick={0,1,2}, extra y ticks=0,
                major grid style={dotted, draw=black!50},
            ]
                \draw[vector={xdarkred}] (0,0) -- (2,0) node[pos=1.0, below] {$\ebc{1}=2\eb{1}$};
                \draw[vector={xdarkblue}] (0,0) -- (1,0) node[below] {$\eb{1}$};
                \draw[vector={xdarkblue}] (0,0) -- (0,1) node[pos=1.2] {$\eb{2}$};
                \draw[vector={xdarkgreen}] (0,0) -- (1,1) node[pos=1.2] {$\vec{v}$};
            \end{axis}
        \end{tikzpicture}
    \end{center}
    \caption{A vector $\vgreen$ and two different sets of basis vectors: in blue the standard basis vectors and in red the basis set $\newB=\left\{2\ebr{1},\ebr{2}\right\}$.}
    \label{fig:vec_in_2_basis_sets}
\end{marginfigure}

We see that by scaling a basis vector by $2$, its repspective component is scaled by $\frac{1}{2}$. In general, when we scale the basis vector $i$ by a non-zero scalar $\alpha$, the respective $i$-th component is scaled by $\frac{1}{\alpha}$.

What about rotating the basis set? Again, we take the vector from \autoref{fig:vec_in_2_basis_sets}, but this time we rotate the standard basis set by some angle $\theta$ (\autoref{fig:vec_in_2_basis_sets_2}). The result is that the vector $\vgreen$ is rotated by $-\theta$ with respect to the new basis set. If you don't see this, look at the angle between $\vgreen$ and $\ebr{1}$: it is exactly $\ang{45}$ (since $\vgreen=\ebr{1}+\ebr{2}$). The angle between $\vgreen$ and $\ebcr{1}$, on the other hand, is smaller than $\ang{45}$ - which means that $\vgreen$ got relatively rotated \textit{towards} it. This is the opposite direction of the rotation from $\ebr{1}$ to $\ebcr{1}$.

\begin{marginfigure}[0\baselineskip]
    \begin{center} 
        \begin{tikzpicture}
            \begin{axis}[
                xynoaxes,
                width=6cm, height=6cm,
                xmin=-0.75, xmax=2.5,
                ymin=-0.75, ymax=2.5,
                xtick={0,1,2}, extra x ticks=0,
                ytick={0,1,2}, extra y ticks=0,
                major grid style={dotted, draw=black!50},
            ]
                \pgfmathsetmacro{\th}{30}
                \draw[opacity=0.5, thick, -stealth, densely dotted] (-0.15,1.2) arc (100:{100+\th-10}:1.15);
                \draw[opacity=0.5, thick, -stealth, densely dotted] (1.15,0.15) arc (10:{\th-5}:1.15);
                \draw[vector={xdarkblue}] (0,0) -- (1,0) node[pos=1.2] {$\eb{1}$};
                \draw[vector={xdarkblue}] (0,0) -- (0,1) node[pos=1.2] {$\eb{2}$};
                \draw[vector={xdarkred}] (0,0) -- ({cos(\th)},{sin(\th)}) node[pos=1.15] {$\ebc{1}$};
                \draw[vector={xdarkred}] (0,0) -- ({sin(-\th)},{cos(\th)}) node[pos=1.15] {$\ebc{2}$};
                \draw[vector={xdarkgreen}] (0,0) -- (1,1) node[pos=1.2] {$\vec{v}$};
            \end{axis}
        \end{tikzpicture}
    \end{center}
    \caption{A vector $\vgreen$ and two different sets of basis vectors: in blue the standard basis vectors and in red the basis set $\newB=\left\{2\ebr{1},\ebr{2}\right\}$.}
    \label{fig:vec_in_2_basis_sets_2}
\end{marginfigure}

It seems that vector coordinates transform \textit{counter} to the change in basis. To be sure, let us take a look at the most general case. Given any vector $\vgreen$ we can express it in a basis set $\oldB=\left\{\ebr{1},\ebr{2},\dots,\ebr{n}\right\}$ using the respective components $\color{xblue}v_{i}\color{black}$:
\begin{equation}
    \vgreen = \displaystyle\sum_{i=1}^{n}\color{xblue}v_{i}\ebr{i}.
    \label{eq:vector_in_old_basis}
\end{equation}
In some other basis $\newB=\left\{\ebcr{1},\ebcr{2},\dots,\ebcr{n}\right\}$ it has other components which we denote as $\color{xred}\tilde{v}_{i}\color{black}$:
\begin{equation}
    \vgreen = \displaystyle\sum_{j=1}^{n}\color{xred}\tilde{v}_{j}\ebcr{j}.
    \label{eq:vector_in_new_basis}
\end{equation}
We can subtitute into \autoref{eq:vector_in_old_basis} the explicit form of the vectors in $\oldB$ (\autoref{eq:coords_BACK_trans_per_vector}):
\begin{equation}
    \begin{aligned}
        \vgreen &= \displaystyle\sum_{i=1}^{n}\color{xblue}v_{i}\ebr{i} \nonumber\\
                &= \displaystyle\sum_{i=1}^{n}\color{xblue}v_{i}
                   \color{black}\left(\displaystyle\sum_{j=1}^{n}F^{-1}_{ji}\ebcr{j}\right) \nonumber\\
                &= \displaystyle\sum_{i=1}^{n}\tikzmark{vstart}{}\displaystyle\sum_{j=1}^{n}\left(F^{-1}_{ji}\color{xblue}v_{i}\tikzmark{vend}{}\color{black}\right)\ebcr{i}.
    \end{aligned}
    \label{eq:transformation_of_components}
\end{equation}
\tikz[overlay, remember picture]{
    \draw[curly={xred}{5pt}{9pt}] ($(pic cs:vstart)+(0,-5pt)$) -- ($(pic cs:vend)+(0,-5pt)$) node[midway, below, yshift=-15pt] {$\tilde{v}_{i}$}; 
}

Comparing the last equality in \autoref{eq:transformation_of_components} to \autoref{eq:vector_in_old_basis}, we see that under a basis change the components of $\vgreen$ go from $\color{xblue}v_{i}\color{black}$ to $\color{xred}\tilde{v}_{i}\color{black}$ via the \textit{inverse transformation} $\Backw$. This is in agreement with what we saw in the specific example: the inverse transformation of scaling is scaling by the inverse, and the inverse transformation of rotation by an angle is rotation by the opposite angle.

Due to this behaviour we say that vectors are \textit{contravarient}, and sometimes even refer to them as \textit{contravarient vectors}\sidenote{this is a hint that there is at least another type of vectors: those that transform together with a change in basis, or \textit{covarient vectors}. Indeed, we will meet these \enquote{co-vectors} soon.}

\subsection{Index position and the Einstein summation convention}
To always remember that vector components are contravarient in regards to change of basis, we will from now on denote them using a superscript instead of a subscript:
\begin{equation}
    \vec{v} = \displaystyle\sum_{i=1}^{n}v^{i}\eb{i}.
    \label{eq:superscript_indeces}
\end{equation}

In the case of a generic vector in $\Rs[n]$ the upper-index notation translates into writing its explicit column-component form as follows:
\begin{equation}
    \vec{v} = \GenericColVec{v}
    \label{eq:upper_index_colvec}
\end{equation}
It is important to always keep in mind that using the upper-index notation for vector components means that powers have to be more explicitely written, as to not confuse the reader. For example, the standard $L_{2}$ norm is written as
\begin{equation}
    \vnorm{v} = \sqrt{\left(v^{1}\right)^{2}+\left(v^{2}\right)^{2}+\dots+\left(v^{n}\right)^{2}}.
    \label{eq:L2_norm_upper_index_notation}
\end{equation}
While this can seem as unnessecary complication now, it will soon become clear why it is needed - especially when we use multiple components of different types (some contravarient and some covarient).

In addition to the upper index notation for vector components, this book (and almost all relevant texts) use the \textit{Einstein summation convention}: where an index appears twice in a single term - once as an upper index and once as a lower index (the order does not matter), we should view the term as having a sum infront of it. This sum is done over all relevant values of the index in question (unless otherwise stated). Below are some examples of using this convention.

\begin{example}{Einstein summation convention \#1}{esc1}
    The standard way of writing a vector explicitely in the basis set $B=\left\{\eb{1},\eb{2},\dots,\eb{n}\right\}$ is
    \begin{equation}
        \vec{v} = \displaystyle\sum_{i=1}^{n}v^{i}\eb{i}.
        \label{eq:std_vector_sum}
    \end{equation}
    Using Einstein's convention, this becomes simply
    \begin{equation}
        \vec{v} = v^{i}\eb{i},
        \label{eq:einsten_vec}
    \end{equation}
    where from context we know that $i\in\left\{1,2,\dots,n\right\}$.
\end{example}

\begin{example}{Einstein summation convention \#2}{esc2}
    The inner product of two vectors $\vec{v},\vec{w}$ can be written explicitely as
    \begin{equation}
        \inner{\vec{v}}{\vec{w}} = \displaystyle\sum_{i=1}^{n}v^{i}w^{i}.
        \label{eq:inner_prod_std}
    \end{equation}
    (note that the upper-index notation is used here)

    However, if we consider the vectors as being matrices, the first being an $1\times n$ matrix and the second an $n\times1$ matrix, we can transpose the first vector from standard column form
    \[
        \vec{v} = \GenericColVec{v}
    \]
    into the row form:
    \[
        \vec{v}^{\top} = \GenericRowVec{v}.
    \]
    (note that the components of $\vec{v}$ as a row-vector are in lower-index notation. The reason for this will become clearer in the next section)
    Indeed, the matrix product of a $1\times n$ matrix with an $n\times1$ matrix has dimenstion of $1\times1=1$, just like a scalar.

    Using Einstein's convention, we can write the inner product as
    \begin{equation}
        \inner{\vec{v}}{\vec{w}} = \GenericRowVec{v}\GenericColVec{w} = v_{i}w^{i}.
        \label{eq:label}
    \end{equation}
    where, again, we know that $i\in\left\{1,2,\dots,n\right\}$ from the context.
\end{example}

\begin{example}{Einstein summation convention \#3}{esc3}
    The product of an $m\times n$ matrix $A$ and a vector $\vec{v}$ can be written in component form as
    \begin{equation}
        \left(A\vec{v}\right)^{i} = \displaystyle\sum_{j=1}^{n}A_{ij}v^{j}.
        \label{eq:label}
    \end{equation}
    \ldots FINISH WRITING THIS CONFUSING PART
\end{example}
