\section{Change of Coordinates}
\newthought{In introductory linear algebra courses} you should have learned about change of coordinate systems: a coordinate system is just another name for a basis set of whatever vector space is used (in this section it's $\Rs[n]$). A change of coordinate system is the transformation of vectors from being represented in one basis set $B=\left\{\eb{1},\eb{2},\dots,\eb{n}\right\}$ to being represented in another basis set $\tilde{B}=\left\{\ebc{1},\ebc{2},\dots,\ebc{n}\right\}$. Since such transformations are linear they are commonly represented in a matrix form.

In this section we will discuss \textit{how} vectors and their components change under change of basis sets. There are many components involved in these kind of transformations, which causes them to be quite confusing. I will therefore color code the equations consistently as a visual guide. In addition, I will always introduce the $\Rs[2]$ case first, before giving the generalized form for $\Rs[n]$.

\subsection{Change of basis set in $\Rs[2]$}
Suppose we use the standard basis set to represent $\Rs[2]$:
\begin{equation}
    \color{xdarkblue}
    B = \left\{\eb{1},\eb{2}\right\} = \left\{\colvec{1;0},\colvec{0;1}\right\},
    \color{black}
    \label{eq:std_basis_set_R2}
\end{equation}
and we want to change our coordinate system to use the following basis set:
\begin{equation}
    \color{xdarkred}
    \tilde{B} = \left\{\ebc{1},\ebc{2}\right\} = \left\{\colvec{2;1},\colvec{-\frac{1}{2};\frac{1}{4}}\right\}.
    \color{black}
    \label{eq:transformed_basis_set_R2}
\end{equation}
(the two basis sets are shown in \cref{fig:two_basis_sets_R2})

\begin{marginfigure}[0\baselineskip]
    \begin{center} 
        \begin{tikzpicture}
            \begin{axis}[
                xynoaxes,
                width=6cm, height=6cm,
                xmin=-0.75, xmax=2.5,
                ymin=-0.75, ymax=2.5,
                xtick={0,1,2}, extra x ticks=0,
                ytick={0,1,2}, extra y ticks=0,
                major grid style={dotted, draw=black!50},
            ]
                \draw[vector={xdarkblue}] (0,0) -- (1,0) node[pos=1.2] {$\eb{1}$};
                \draw[vector={xdarkblue}] (0,0) -- (0,1) node[pos=1.2] {$\eb{2}$};
                \draw[vector={xdarkred}] (0,0) -- (2,1) node[pos=1.075] {$\ebc{1}$};
                \draw[vector={xdarkred}] (0,0) -- (-0.5,0.25) node[pos=1.3] {$\ebc{2}$};
            \end{axis}
        \end{tikzpicture}
    \end{center}
    \caption{The standard basis set $\color{xdarkblue}{B}$ and a new basis set $\color{xdarkred}{\tilde{B}}$ shown together.}
    \label{fig:two_basis_sets_R2}
\end{marginfigure}

\vspace{1.5em}
\begin{equation}
    \Forw =
    \begin{bmatrix}
        \tikzmark{ebc1}{2} & \tikzmark{ebc2}{-\frac{1}{2}} \\
        1 & \frac{1}{4}
    \end{bmatrix}.
    \label{eq:forward_trans}
\end{equation}

\tikz[overlay, remember picture]{
    \node[xdarkred] (ebc1txt) at ($(pic cs:ebc1)+(4pt,1cm)$) {$\ebc{1}$};
    \node[xdarkred] (ebc2txt) at ($(pic cs:ebc2)+(4pt,1cm)$) {$\ebc{2}$};
    \draw[-stealth, xdarkred] (ebc1txt) -- ($(pic cs:ebc1)+(4pt,10pt)$);
    \draw[-stealth, xdarkred] (ebc2txt) -- ($(pic cs:ebc2)+(4pt,10pt)$);
}

Now, if we want to transform $\ebc{1}$ and $\ebc{2}$ into $\ebcr{1}$ and $\ebcr{2}$ using $\Forw$, we simply multiply them by $\Forw$:
\begin{align}
    \ebcr{1} &= \Forw \ebr{1},\\\nonumber
    \ebcr{2} &= \Forw \ebr{2}.
    \label{eq:transforming_ebs}
\end{align}

To make \cref{eq:transforming_ebs} more concise, we can collect the two vectors into a matrix disguised as a row vecor:
\begin{equation}
    \color{xdarkblue}
    \eb{} =
    \begin{bmatrix}
        1 & 0\\
        0 & 1
    \end{bmatrix} = \rowvec{\eb{1},\eb{2}}
    \color{black}.
    \label{eq:eb_matrix}
\end{equation}

We then get that \cref{eq:transforming_ebs} can be written in vector-matrix notation as
\begin{equation}
    \color{xdarkred}\ebcr{} = \rowvec{\eb{1};\eb{2}}
    \color{black} =
    \color{xdarkblue}\rowvec{\eb{1};\eb{2}}
    \color{black}
        \begin{bmatrix}
            2 & -\frac{1}{2}\\
            1 & \frac{1}{4}
        \end{bmatrix}
        = \rowvec{2\ebr{1}+\ebr{2};-\frac{1}{2}\ebr{1}+\frac{1}{4}\ebr{2}}.
    \label{eq:full_forward_trans}
\end{equation}

The reverse transformation can be calculated by applying the inverse transformation $\Backw$ on \cref{eq:full_forward_trans}:
\begin{equation}
    \color{xdarkred}\rowvec{\ebc{1};\ebc{2}}
    \color{black}\Backw =
    \left(
        \color{xdarkblue}\rowvec{\eb{1};\eb{2}}
        \color{black} \Forw
    \right)\Backw =
    \color{xdarkblue}\rowvec{\eb{1};\eb{2}}.
    \label{eq:eb_from_ebc}
\end{equation}
(where $\Backw=\begin{bmatrix}\frac{1}{4} & \frac{1}{2}\\ -1 & 2\end{bmatrix}$)

\subsection{Contravarience behabiour of vectors}
