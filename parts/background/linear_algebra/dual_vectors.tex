\section{Dual Vectors and Dual Spaces}
% Overview:
% Measuring vectors: rulers. How they look like in R2, R3, etc.
% Every ruler can be represented using a specific vector in Rn (direction + density). The measurement is then done via the inner product.
% These inner products actually represent all possible linear functionals Rn->R. They make a linear space on their own.
% Define general idea of dual spaces.
% Basis sets and change of basis - covarience, contravarience and all that.
% relevant SE answers: https://math.stackexchange.com/questions/3749/why-do-we-care-about-dual-spaces

\subsection{Linear measurements and rulers\\(or: why should I care about dual vectors?)}
\begin{marginfigure}
    \begin{center}
        \begin{tikzpicture}
            \begin{axis}[
                xyplane,
                width=7cm, height=7cm,
            ]
            % \draw[vector={xblue}] (0,0) -- (2,-3);
            \addplot[name path=A, function={xred}] {6*exp(-x^2/2)-2};
            \addplot[name path=B, function={xblue}] {5/(1+exp(-x))};
            \addplot[xpurple, fill opacity=0.2] fill between[of=A and B, soft clip={domain=.5:5}];
            \addplot[xgreen, fill opacity=0.2] fill between[of=A and B, soft clip={domain=-1:0.5}];
            \end{axis}
        \end{tikzpicture}
    \end{center}
    \caption{Test.}
    \label{fig:test}
\end{marginfigure}


To be written:

\subsection{Introducing some formalism}
\begin{enumerate}
    \item Dual vectors form a vector space $\dualspace{V}$.
    \item Formal definition of dual spaces.
    \item Examples of dual vectors of functions?..
\end{enumerate}

\subsection{Basis sets and coordinate transformations}
\begin{enumerate}
    \item Dual basis: converting from a basis set in $V$ to its dual in $\dualspace{V}$.
    \item Covariance of dual vectors basis change vs. contra-varience of vectors.
\end{enumerate}

% \begin{figure}
%     \begin{center}
%         \tdplotsetmaincoords{50}{110}
%         \begin{tikzpicture}[tdplot_main_coords, rotate=0, scale=0.85]
%             \pgfmathsetmacro{\a}{1.5}
%             \pgfmathsetmacro{\b}{\a+0.5}
%             \foreach \z in {-2,-1.25,...,2}
%                 \draw[very thick, xblue, fill=xblue, opacity=0.5] (-\a,-\a,\z) -- (-\a,\a,\z) -- (\a,\a,\z) -- (\a,-\a,\z) -- cycle;
%             \draw[vector={xdarkblue}] (-\b,\b,-1.5) -- (-\b,\b,1.5);
%         \end{tikzpicture}
%         \hfill
%         \tdplotsetmaincoords{40}{120}
%         \begin{tikzpicture}[tdplot_main_coords, rotate=105, scale=0.85]
%             \pgfmathsetmacro{\a}{1.5}
%             \pgfmathsetmacro{\b}{\a+0.5}
%             \foreach \z in {-3,-1.5,...,3}
%                 \draw[very thick, xdarkgreen, fill=xgreen, opacity=0.5] (-\a,-\a,\z) -- (-\a,\a,\z) -- (\a,\a,\z) -- (\a,-\a,\z) -- cycle;
%             \draw[vector={xdarkgreen}] (-\b,\b,-1) -- (-\b,\b,1);
%         \end{tikzpicture}
%     \end{center}
%     \caption{Representation of rulers in $\Rs[3]$; as with the lines in the case of $\Rs[2]$, I added a vector showing the direction and density of the graduation marks. Note how the planes (stacks) in blue are more densely spaced compared to the green planes, and therefore the vector representing that respective ruler is longer compared to the vector representing the ruler in green.}
%     \label{fig:rulers_3D}
% \end{figure}
