\section{Dual Vectors and Dual Spaces}
% Overview:
% Measuring vectors: rulers. How they look like in R2, R3, etc.
% Every ruler can be represented using a specific vector in Rn (direction + density). The measurement is then done via the inner product.
% These inner products actually represent all possible linear functionals Rn->R. They make a linear space on their own.
% Define general idea of dual spaces.
% Basis sets and change of basis - covarience, contravarience and all that.
% relevant SE answers: https://math.stackexchange.com/questions/3749/why-do-we-care-about-dual-spaces

\subsection{Measurements and rulers}
\newthought{Usually, dual vectors are taught} by hitting the students with the definition of a dual space and then analyzing its properties. It's all very abstract and often leaves the students with a constant question in mind: \enquote{why do we care about dual vectors?}

I would like to take a different approach here: instead of confronting you with the definition and then discuss practical details, I will start with explaining \textit{why} we care about dual vectors in the first place.

Let us begin with discussing rulers\sidenote{The idea for this approach comes from a beautiful answer by \textit{Aloizio Macedo} to \href{https://math.stackexchange.com/questions/3749/why-do-we-care-about-dual-space}{a question in the mathematics stack exchange website}.}. A ruler is essentially a geometric object which allows one to measure the \textit{lengths} of different objects by counting the number of graduation lines between the beginning and end of an object.

\begin{figure}
    \begin{center}
        \begin{tikzpicture}
            \draw[thick] (0.5,0) rectangle (10,1.25);
            \foreach \x [count=\k from 0] in {1,2,...,9} 
                \draw[thick] (\x, 1.25) -- ++(0.0,-0.5) node[below] (g\k) {$\k$};
            \foreach \y in {1.5,2.5,...,8.5} 
                \draw[thick] (\y, 1.25) -- ++(0.0,-0.4);
            \foreach \z in {1.1,1.2,...,8.9} 
                \draw[thick] (\z, 1.25) -- ++(0.0,-0.25);
            \node[below of=g0, font=\tiny, yshift=20] {unit};
            \draw[thick, densely dashed, black!50] (3.5,1.5) -- ++(0,-0.25);
            \draw[thick, densely dashed, black!50] (6.5,1.5) -- ++(0,-0.25);
            \draw[vector={xred}] (3.5,1.5) -- ++(3,0);
        \end{tikzpicture}
    \end{center}
    \caption{Measuring a vector using a ruler: the start of the vector sits at $2.5$ units, while its head is at $5.5$ units. Therefore we say that the vector is $5.5-2.5=3$ units in length.}
    \label{fig:ruler_measure}
\end{figure}
