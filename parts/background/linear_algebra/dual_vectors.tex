\section{Dual Vectors and Dual Spaces}
% Overview:
% Measuring vectors: rulers. How they look like in R2, R3, etc.
% Every ruler can be represented using a specific vector in Rn (direction + density). The measurement is then done via the inner product.
% These inner products actually represent all possible linear functionals Rn->R. They make a linear space on their own.
% Define general idea of dual spaces.
% Basis sets and change of basis - covarience, contravarience and all that.
% relevant SE answers: https://math.stackexchange.com/questions/3749/why-do-we-care-about-dual-spaces

\subsection{Measurements and rulers}
\newthought{Usually, dual vectors are taught} by hitting the students with the definition of a dual space and then analyzing its properties. It's all very abstract and often leaves the students with a constant question in mind: \enquote{why do we care about dual vectors?}

I would like to take a different approach here: instead of confronting you with the definition and then discuss practical details, I will start with explaining \textit{why} we care about dual vectors in the first place.

Let us begin with discussing rulers\sidenote{The idea for this approach comes from a beautiful answer by \textit{Aloizio Macedo} to \href{https://math.stackexchange.com/questions/3749/why-do-we-care-about-dual-space}{a question in the mathematics stack exchange website}.}. A ruler is essentially a geometric measuring tool which allows one to measure the \textit{lengths} of different objects by counting the number of graduation lines between the beginning and end of an object (\cref{fig:ruler_measure}). Of course in linear algebra, the geometric objects we measure using rulers are vectors.

\begin{figure}
    \begin{center}
        \begin{tikzpicture}
            \draw[thick, fill=black!5] (0.5,0) rectangle (10,1.25);
            \foreach \x [count=\k from 0] in {1,2,...,9} 
                \draw[thick] (\x, 1.25) -- ++(0.0,-0.5) node[below] (g\k) {$\k$};
            \foreach \y in {1.5,2.5,...,8.5} 
                \draw[thick] (\y, 1.25) -- ++(0.0,-0.4);
            \foreach \z in {1.1,1.2,...,8.9} 
                \draw[thick] (\z, 1.25) -- ++(0.0,-0.25);
            \node[below of=g0, font=\tiny, yshift=20] {unit};
            \draw[thick, densely dashed, black!50] (3.5,1.5) -- ++(0,-0.25);
            \draw[thick, densely dashed, black!50] (6.5,1.5) -- ++(0,-0.25);
            \draw[vector={xred}] (3.5,1.5) -- ++(3,0);
        \end{tikzpicture}
    \end{center}
    \caption{Measuring a vector using a ruler: the start of the vector sits at $2.5$ units, while its head is at $5.5$ units. Therefore we say that the vector is $5.5-2.5=3$ units in length.}
    \label{fig:ruler_measure}
\end{figure}

We can represent a ruler in $\Rs[2]$ as follows: the $0$-graduation, which I denote as $L_{0}$ here, is represented by a line going through the origin with a direction orthogonal to the that of the ruler. The rest of the graduation marks are then drawn as the lines that are parallel to $L_{0}$ such that they are spaced equaly apart (see \cref{fig:rulers_2D}). To mark the direction of the ruler more clearly, a vector with the same direction of the ruler is added at the origin, such that its length corresponds to the \enquote{density} of the graduation marks.

\begin{figure}
    \begin{center}
        \begin{tikzpicture}
            \begin{axis}[
                xynogrid,
                width=6cm, height=6cm,
                xmin=-4, xmax=4,
                ymin=-4, ymax=4,
                ticks=none,
                axis on top,
                name=leftGraph,
            ]
            \rulerTwoD{1}{3}{.7}{4}{xblue}{blue!75!black}

            \end{axis}
            \begin{axis}[
                xynogrid,
                width=6cm, height=6cm,
                xmin=-4, xmax=4,
                ymin=-4, ymax=4,
                ticks=none,
                axis on top,
                name=rightGraph,
                at={($(leftGraph.south east)+(1.5cm,0)$)},
            ]
            \rulerTwoD{-2}{1}{0.9}{10}{xorange}{orange!75!black}
            \end{axis}
        \end{tikzpicture}
    \end{center}
    \caption{Two rulers in $\Rs[2]$, each represented as the infinite set of lines which are parallel to the zero graduation $L_{0}$, and are equaly spaced apart. Note the vector drawn at the origin of each figure: it has the same direction as the respective ruler, and its length corresponds to the \enquote{density} of the graduation marks - the more dense the graduation marks, the longer the vector.}
    \label{fig:rulers_2D}
\end{figure}

In $\Rs[3]$ the graduations are represented in the same way, except that \textit{planes} are used instead of line (figure).

\begin{figure}
    \begin{center}
        \tdplotsetmaincoords{50}{110}
        \begin{tikzpicture}[tdplot_main_coords, rotate=0, scale=0.85]
            \pgfmathsetmacro{\a}{1.5}
            \pgfmathsetmacro{\b}{\a+0.5}
            \foreach \z in {-2,-1.25,...,2}
                \draw[very thick, xblue, fill=xblue, opacity=0.5] (-\a,-\a,\z) -- (-\a,\a,\z) -- (\a,\a,\z) -- (\a,-\a,\z) -- cycle;
            \draw[vector={xdarkblue}] (-\b,\b,-1.5) -- (-\b,\b,1.5);
        \end{tikzpicture}
        \hfill
        \tdplotsetmaincoords{40}{120}
        \begin{tikzpicture}[tdplot_main_coords, rotate=105, scale=0.85]
            \pgfmathsetmacro{\a}{1.5}
            \pgfmathsetmacro{\b}{\a+0.5}
            \foreach \z in {-3,-1.5,...,3}
                \draw[very thick, xdarkgreen, fill=xgreen, opacity=0.5] (-\a,-\a,\z) -- (-\a,\a,\z) -- (\a,\a,\z) -- (\a,-\a,\z) -- cycle;
            \draw[vector={xdarkgreen}] (-\b,\b,-1) -- (-\b,\b,1);
        \end{tikzpicture}
    \end{center}
    \caption{Representation of a ruler in $\Rs[3]$.}
    \label{fig:ruler_measure_at_angle}
\end{figure}
