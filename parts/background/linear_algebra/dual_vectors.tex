\section{Dual Vectors and Dual Spaces}
% Section overview:
% * important set of linear transformations: R^n -> R. Can provide us a kind of "measurement" of vectors.
% * We call the space of all such LTs "the dual space" of R^n.
% * Show the "layers" interpretation of dual vectors.
% * Any dual vector can be represented using the dot product and elements of R^n* can be represented by a list of n real numbers... this is actually R^n itself!
% If we represent the elements of R^n* as row vectors, we get a nice completness similar to matrix-matrix product.

% relevant SE answers: https://math.stackexchange.com/questions/3749/why-do-we-care-about-dual-spaces

\newthought{An important aspect of vector spaces}, which is sometimes waved away, is the question of \textit{measurement}: how do we give vectors a sense of magnitude? By \textit{magnitude} I mean a single real number we assign to each vector. Well, there's one obvious way: the norm. Normally\sidenote{pun kind-of intended.} it is either given as is (due to the Pythagorean theorem), or in the case of more abstract vector spaces formalized as the square root of the inner product of a vector with itself, i.e.
\begin{equation}
    \vnorm{v} = \sqrt{\inner{\vec{v}}{\vec{v}}}.
    \label{eq:vector_norm}
\end{equation}

Of course, other so-called \enquote{$p$-norms} are possible and often used:
\begin{equation}
    \norm{\vec{v}}_{p} = \left(\abs{v_{1}}^{p}+\abs{v_{2}}^{p} + \dots + \abs{v_{n}}^{p}\right)^{\frac{1}{p}},
    \label{eq:label}
\end{equation}
where $p=2$ is the normal \textit{Euclidean} norm we're used to.

However, with the exception of the case $p=1$, these norms are non-linear. And if there's one insight that should be very clear to anyone who went through some university-level mathematics, it is that linear structures are so much easier to deal with than almost anything else\sidenote{that's why linear approximations are so often used all throughout science}.

So instead of using the norm as a measurement, we would ideally like to use some linear function to measure our vectors. For example, let's try to come up with a linear way to measure vectors in $\Rs[3]$: given a vector $\vec{v}$, we can derive a linear map $\phi:\Rs[3]\to\Rs$ as follows:
\begin{equation}
    \phi\left(\vec{v}\right) = 3v_{1} - v_{2} + 5v_{3}.
    \label{eq:linear_map_R3_example}
\end{equation}
Let's apply $\phi$ to some vectors and see the results:
\begin{align*}
    \colvec{1;0;0} &\to 3\cdot1 -1\cdot0 + 5\cdot0 = 3.\\
    \colvec{0;1;0} &\to 3\cdot0 -1\cdot1 + 5\cdot0 = -1.\\
    \colvec{0;0;1} &\to 3\cdot0 -1\cdot0 + 5\cdot1 = 5.\\
    \colvec{1;-2;-1} &\to 3\cdot1 -1\cdot(-2) + 5\cdot(-1) = 3+2-5=0.\\
\end{align*}
We see that using this specific $\phi$ the standard basis vectors are \enquote{measured} to be of different values, and some vectors like $\vec{v}=\colvec{1;-2;-1}$ can measured to be $0$.

A bit of thinking further shows that in fact, \textit{any} linear map of the form $\phi:\Rs[3]\to\Rs$ we can apply to the vectors of $\Rs[3]$ has to be in the following form:
\begin{equation}
    \phi\left(\vec{v}\right) = \alpha_{1}v_{1} + \alpha_{2}v_{2} + \alpha_{3}v_{3},
    \label{eq:general_linear_measure_R3}
\end{equation}
where $\alpha_{1},\alpha_{2},\alpha_{3}$ are some real numbers. But this is exactly the definition of the inner product between $\colvec{\alpha_{1};\alpha_{2};\alpha_{3}}$ and $\vec{v}$! So we see that \textit{any} linear measurement on $\Rs[3]$ will take the form ofan inner product with some given vector.

In other words, any vector in $\Rs[3]$ can be used to linearly measure all vectors in $\Rs[3]$ via the inner product, and in this context act as a \textit{functional} on elements of the space. Moreover, any possible linear measurement of vectors in $\Rs[3]$ would be of this form - and thus the space of linear maps of the form $\Rs[3]\to\Rs$ is itself a vector space with the same structure as $\Rs[3]$\sidenote{In fact, \textit{it is} $\Rs[3]$ in any sensible meaning.} and in this context we call it the \textit{dual space} of $\Rs[3]$, and its elements \textit{dual vectors}.

Now, of course choosing $\Rs[3]$ was pretty arbitrary, so let us generalize somewhat the defintions:
\begin{definition}{Text}{deftext}
    The dual space of $\Rs[n]$ is the space of all linear maps of the form
    \begin{equation*}
        \phi:\Rs[n]\to\Rs,
    \end{equation*}
    which can be represented as the inner product between the elements of $\Rs[n]$. 

    We call the elements of the dual space \textit{dual vectors}, and denote them with an asterisk: $\dualvec{v}$.
\end{definition}

To distinguish between vectors and dual vectors in practice, we represent dual vectors as \textit{row-vectors} instead of the usual \textit{column-vector} for vectors. For example, a dual vector from $\Rs[3]$ will have the form
\begin{equation}
    \dualvec{a} = \rowvec{a_{1};a_{2};a_{3}}.
    \label{eq:dual_vectors_row_form}
\end{equation}

This way, the action of dual vectors on vectors (as functionals) is clearly seen via writing the inner product: given the dual vector $\dualvec{a}\in\Rs[n]$ and the vector $\vec{v}\in\Rs[n]$, their inner product can be written explicitely as
\begin{equation}
    \inner{\dualvec{a}}{\vec{v}} = \GenericRowVec{a} \GenericColVec{v} = a_{1}v_{1} + a_{2}v_{2} + \dots + a_{n}v_{n}.
    \label{eq:dualvec_vec_inner}
\end{equation}
This corresponds well to a product between two matrices, one of dimensions $1\times n$ and the other of dimensions $n\times1$ - which will become handy when we discuss \textit{multi-linear forms}\sidenote{yes, these things get more complicated\ldots}.

\begin{note}{Product order}{}
    Using this formulation we must always multiply the dual vector and the vector such that the dual vector is \textit{on the left}. Otherwise, we should expect to get a matrix of dimensions $n\times n$.
\end{note}

A way to visualize dual vectors is using \textit{stacks}: since dual vectors have the same representation as \enquote{usual} vectors, they define a direction in a geometrical $n$-dimensional space. Since each direction in $\Rs[n]$ has a single corresponding $n-1$-dimensional hyperplane, to differentiate vectors and dual vectors we visualize dual vectors using the hyperplanes.

This is a bit abstract, so let us look at our two favourite examples: $\Rs[2]$ and $\Rs[3]$. In $\Rs[2]$ the $n-1$-dimensional hyperplanes are simply lines (since $2-1=1$), which \enquote{stack} on top of each other in the direction of the dual vector they represent. See \cref{fig:2d_stacks_1,fig:2d_stacks_2} for a graphical representation.

\begin{marginfigure}
    \begin{center}
        \begin{tikzpicture}
            \begin{axis}[
                xynogrid,
                width=6cm, height=6cm,
            ]
            \foreach \dy in {-5,...,5}{
                \addplot[hyperplane1D={xpurple}] {-x+2*\dy};
            }
            \draw[vector={xpurple!75!black}, thick] (0,0) -- (1.2,1.2);
            \end{axis}
        \end{tikzpicture}
    \end{center}
    \caption{Graphical representation of the dual vector $\dualvec{a}=\rowvec{1;1}$. The purple lines are 1-dimensional hyperplanes, with one intersecting the origin. The small arrow shows the vector $\vec{v}=\colvec{1;1}$ which is orthogonal to the hyperplanes.}
    \label{fig:2d_stacks_1}
\end{marginfigure}
\begin{marginfigure}
    \begin{center}
        \begin{tikzpicture}
            \begin{axis}[
                xynogrid,
                width=6cm, height=6cm,
            ]
            \foreach \dy in {-6,...,6}{
                \addplot[hyperplane1D={xblue}] {5/6*\dy};
            }
            \draw[vector={xblue!75!black}, thick] (1,0) -- (1,4);
            \end{axis}
        \end{tikzpicture}
    \end{center}
    \caption{A more \enquote{dense} stack oriented in parallel to the $x$-axis. This represents the dual vector $\dualvec{b}=\rowvec{}$, which has a greater magnitude than $\dualvec{a}$.}
    \label{fig:2d_stacks_2}
\end{marginfigure}
