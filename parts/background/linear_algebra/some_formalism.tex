\section{Some Formalism and General Vector Spaces}
\newthought{Up until now in this chapter} we only looked at vectors in $\Rs[n]$ (and usually with $n\in\left\{2,3\right\}$). However, the ideas presented can be generalized to all kind of vector spaces. Yes - it is finally time to present the \enquote{mathematician's view of vectors}!

To do so, let us first overview some fundamental (trivial, even) properties of $\Rs[n]$:
\begin{descitemize}
    \item[Closure of vector addition] the sum of any two vectors $\vec{u}$ and $\vec{v}$ in $\Rs[n]$ is also a vector in $\Rs[n]$ - i.e. 
    \begin{equation}
        \text{if}\ \vec{u}+\vec{v}=\vec{w},\ \text{then}\ \vec{w}\in\Rs[n].
        \label{eq:vector_addition_closure}
    \end{equation}
    
    \item[Commutativity of vector addition] resulting from the parallelogram rule, the addition of vectors is commutative - i.e.
    \begin{equation}
        \vec{u}+\vec{v}=\vec{v}+\vec{u}.
        \label{eq:vector_addition_commutative_2}
    \end{equation}

    \item[Associativity of vector addition] the order of adding multiple vectors does not matter: for any three vectors $\vec{u},\ \vec{v},\ \vec{w}$ in $\Rs[n]$,
    \begin{equation}
        \vec{u}+\left(\vec{v}+\vec{w}\right) = \left(\vec{u}+\vec{v}\right)+\vec{w}.
        \label{eq:vector_addition_associative}
    \end{equation}
    
    \item[Existence of zero] the zero vector $\vec{0}$ is a neutral to addition - i.e.
    \begin{equation}
        \forall \vec{u}\in\Rs[n]:\ \vec{u}+\vec{0}=\vec{0}+\vec{u}=\vec{u}.
        \label{eq:vector_zero_existence}
    \end{equation}

    \item[Existence additive inverse] for any vector $\vec{v}\in\Rs[n]$ there's an inverse - i.e
    \begin{equation}
        \forall \vec{v}\in\Rs[n]:\ \exists\left(-\vec{v}\right), \vec{v}+\left(-\vec{v}\right) = \vec{0}.
        \label{eq:vector_zero_addition}
    \end{equation}

    \item[Closure of scalar multiplication] the result of scaling by $\lambda\in\Rs$ of any vector $\vec{v}\in\Rs[n]$ is also in $\Rs[n]$ - i.e. 
    \begin{equation}
        \forall\vec{v}\in\Rs[n] \text{and}\ \forall\lambda\in\Rs:\ \lambda\vec{v}\in\Rs[n].
        \label{eq:scalar_multiplication_closure}
    \end{equation}

    \item[Associativity of scalar multiplication] for any two scalars $\lambda,\mu\in\Rs$, the order of scaling a vector $\vec{v}\in\Rs[n]$ doesn't matter - i.e.
    \begin{equation}
        \left(\lambda\vec{v}\right)\cdot\mu = \lambda\cdot\left(\mu\vec{v}\right).
        \label{eq:scalar_multiplication_associative}
    \end{equation}

    \item[Existnce of unity] the number $1$ is neutral with scaling - i.e.
        \begin{equation}
            \forall \vec{v}\in\Rs[n]: 1\vec{v}=\vec{v}.
            \label{eq:scalar_multiplication_unity}
        \end{equation}

    \item[Distributive laws] vector addition and scaling are distributive together with addition of scalars, i.e. for any $\vec{v},\vec{u}\in\Rs[n]$ and $\lambda,\mu\in\Rs$:
        \begin{align}
            \lambda\left(\vec{v}+\vec{u}\right) &= \lambda\vec{v} + \lambda\vec{u},\ \text{and}\\ \left(\lambda+\mu\right)\vec{v} &= \lambda\vec{v} + \mu\vec{v}.
            \label{eq:label}
        \end{align}
\end{descitemize}

There are many other mathematical objects that have the same properties, for example matrices of the same dimensions, with the usual matrix addition and scalar multiplication operations: in this case, the zero matrix is the neutral element to addition and the scalar $1$ is the unity. Another example using functions can be seen in (EXAMPLE REF).

We use the properties we described to define a general idea of a vector space. First, we choose a \textit{field} $\mathbb{F}$: a set whose elements will be our scalars. Usually the field is either $\Rs$ or $\Cs$, but there are many other options which we will ignore for now. Then, we choose a set $V$ and define two operations: one operation between two elements of $V$ (which we call \textit{addition}) and one between elements of $V$ and elements of the field $\mathbb{F}$ (which we call \text{scalar multiplication}).

Now, if we defined all of the above (field + set + two operations) \textbf{and} they behave according to the properties we described for $\Rs[n]$ (closure, commutativity, associativity, etc.) \textbf{then} we say that the set $V$ is a vector set \textit{over} $\mathbb{F}$ with the two operations.

%-- EXAMPLE: trivial vector space: {-1, 1, 0}?..
\begin{example}{Trivial vector space}{trivial_vector_space}
    What is the smallest vector space we can construct using a subset of $\Rs$ as a field? If we choose $V=\mathbb{F}=\emptyset$ and the usual addition and multiplication between, say, real numbers - this is a valid vector space (check for youself!). However, it's a very boring one. Let's try to build a small vector space with more than one element in $V$: \ldots
\end{example}
