\section{Some Formalism and General Vector Spaces}

%% !!! CORRECT ISSUE with \cref and example environment !!! %%

% Outline:
% * Linear algebra on R^n prooved useful (e.g. linear transformations, basis sets, etc.). It would be nice to have similar results for other structures.
% * To find more structures for which we can derive similar ideas, we first need to overview the fundamental properties of vectors in R^n, and see if there are any other structures with similar properties.
% * Fundamental properties
% * Matrices have the same properties! (+elaboration)
% * Polynomial functions have the same properties! (+elaboration)
% * Functions have the same properties! (+elaboration)
% * Using these properties to define a vector space + short discussion.

% \begin{descitemize}
%     \item[Closure of vector addition] the sum of any two vectors $\vec{u}$ and $\vec{v}$ in $\Rs[n]$ is also a vector in $\Rs[n]$ - i.e. 
%     \begin{equation}
%         \text{if}\ \vec{u}+\vec{v}=\vec{w},\ \text{then}\ \vec{w}\in\Rs[n].
%         \label{eq:vector_addition_closure}
%     \end{equation}
%     
%     \item[Commutativity of vector addition] resulting from the parallelogram rule, the addition of vectors is commutative - i.e.
%     \begin{equation}
%         \vec{u}+\vec{v}=\vec{v}+\vec{u}.
%         \label{eq:vector_addition_commutative_2}
%     \end{equation}
%
%     \item[Associativity of vector addition] the order of adding multiple vectors does not matter: for any three vectors $\vec{u},\ \vec{v},\ \vec{w}$ in $\Rs[n]$,
%     \begin{equation}
%         \vec{u}+\left(\vec{v}+\vec{w}\right) = \left(\vec{u}+\vec{v}\right)+\vec{w}.
%         \label{eq:vector_addition_associative}
%     \end{equation}
%     
%     \item[Existence of zero] the zero vector $\vec{0}$ is a neutral to addition - i.e.
%     \begin{equation}
%         \forall \vec{u}\in\Rs[n]:\ \vec{u}+\vec{0}=\vec{0}+\vec{u}=\vec{u}.
%         \label{eq:vector_zero_existence}
%     \end{equation}
%
%     \item[Existence additive inverse] for any vector $\vec{v}\in\Rs[n]$ there's an inverse - i.e
%     \begin{equation}
%         \forall \vec{v}\in\Rs[n]:\ \exists\left(-\vec{v}\right), \vec{v}+\left(-\vec{v}\right) = \vec{0}.
%         \label{eq:vector_zero_addition}
%     \end{equation}
%
%     \item[Closure of scalar multiplication] the result of scaling by $\lambda\in\Rs$ of any vector $\vec{v}\in\Rs[n]$ is also in $\Rs[n]$ - i.e. 
%     \begin{equation}
%         \forall\vec{v}\in\Rs[n] \text{and}\ \forall\lambda\in\Rs:\ \lambda\vec{v}\in\Rs[n].
%         \label{eq:scalar_multiplication_closure}
%     \end{equation}
%
%     \item[Associativity of scalar multiplication] for any two scalars $\lambda,\mu\in\Rs$, the order of scaling a vector $\vec{v}\in\Rs[n]$ doesn't matter - i.e.
%     \begin{equation}
%         \left(\lambda\vec{v}\right)\cdot\mu = \lambda\cdot\left(\mu\vec{v}\right).
%         \label{eq:scalar_multiplication_associative}
%     \end{equation}
%
%     \item[Existnce of unity] the number $1$ is neutral with scaling - i.e.
%         \begin{equation}
%             \forall \vec{v}\in\Rs[n]: 1\vec{v}=\vec{v}.
%             \label{eq:scalar_multiplication_unity}
%         \end{equation}
%
%     \item[Distributive laws] vector addition and scaling are distributive together with addition of scalars, i.e. for any $\vec{v},\vec{u}\in\Rs[n]$ and $\lambda,\mu\in\Rs$:
%         \begin{align}
%             \lambda\left(\vec{v}+\vec{u}\right) &= \lambda\vec{v} + \lambda\vec{u},\ \text{and}\\ \left(\lambda+\mu\right)\vec{v} &= \lambda\vec{v} + \mu\vec{v}.
%             \label{eq:label}
%         \end{align}
% \end{descitemize}
